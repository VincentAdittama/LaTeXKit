%% ============================================
%% COMPONENT: Code Blocks and Syntax
%% ============================================
%% Usage: Include this file in your preamble
%% %% ============================================
%% COMPONENT: Code Blocks and Syntax
%% ============================================
%% Usage: Include this file in your preamble
%% %% ============================================
%% COMPONENT: Code Blocks and Syntax
%% ============================================
%% Usage: Include this file in your preamble
%% %% ============================================
%% COMPONENT: Code Blocks and Syntax
%% ============================================
%% Usage: Include this file in your preamble
%% \input{components/code.tex}
%% ============================================

\usepackage{listings}
\usepackage{xcolor}

%% Define colors for syntax highlighting
\definecolor{codegreen}{rgb}{0,0.6,0}
\definecolor{codegray}{rgb}{0.5,0.5,0.5}
\definecolor{codepurple}{rgb}{0.58,0,0.82}
\definecolor{codeblue}{rgb}{0.13,0.29,0.53}
\definecolor{codebg}{rgb}{0.95,0.95,0.95}

%% Configure listings style
\lstdefinestyle{default}{
    backgroundcolor=\color{codebg},
    commentstyle=\color{codegreen},
    keywordstyle=\color{codeblue}\bfseries,
    numberstyle=\tiny\color{codegray},
    stringstyle=\color{codepurple},
    basicstyle=\ttfamily\footnotesize,
    breakatwhitespace=false,
    breaklines=true,
    captionpos=b,
    keepspaces=true,
    numbers=left,
    numbersep=8pt,
    showspaces=false,
    showstringspaces=false,
    showtabs=false,
    tabsize=2,
    frame=single,
    rulecolor=\color{codegray}
}

\lstset{style=default}

%% Inline code command
\newcommand{\inlinecode}[1]{%
  \colorbox{codebg}{\texttt{\small #1}}%
}

%% Command: Code block with language
%% Usage: In document, use \begin{lstlisting}[language=Python]
%% Or use the custom environment below

%% Custom code environment
\lstnewenvironment{code}[1][]{%
  \lstset{
    style=default,
    #1
  }
}{}

%% Example usage (commented out):
% \inlinecode{print("Hello, World!")}
%
% \begin{code}[language=Python, caption=Example Python Code]
% def hello_world():
%     print("Hello, World!")
% \end{code}

%% ============================================

\usepackage{listings}
\usepackage{xcolor}

%% Define colors for syntax highlighting
\definecolor{codegreen}{rgb}{0,0.6,0}
\definecolor{codegray}{rgb}{0.5,0.5,0.5}
\definecolor{codepurple}{rgb}{0.58,0,0.82}
\definecolor{codeblue}{rgb}{0.13,0.29,0.53}
\definecolor{codebg}{rgb}{0.95,0.95,0.95}

%% Configure listings style
\lstdefinestyle{default}{
    backgroundcolor=\color{codebg},
    commentstyle=\color{codegreen},
    keywordstyle=\color{codeblue}\bfseries,
    numberstyle=\tiny\color{codegray},
    stringstyle=\color{codepurple},
    basicstyle=\ttfamily\footnotesize,
    breakatwhitespace=false,
    breaklines=true,
    captionpos=b,
    keepspaces=true,
    numbers=left,
    numbersep=8pt,
    showspaces=false,
    showstringspaces=false,
    showtabs=false,
    tabsize=2,
    frame=single,
    rulecolor=\color{codegray}
}

\lstset{style=default}

%% Inline code command
\newcommand{\inlinecode}[1]{%
  \colorbox{codebg}{\texttt{\small #1}}%
}

%% Command: Code block with language
%% Usage: In document, use \begin{lstlisting}[language=Python]
%% Or use the custom environment below

%% Custom code environment
\lstnewenvironment{code}[1][]{%
  \lstset{
    style=default,
    #1
  }
}{}

%% Example usage (commented out):
% \inlinecode{print("Hello, World!")}
%
% \begin{code}[language=Python, caption=Example Python Code]
% def hello_world():
%     print("Hello, World!")
% \end{code}

%% ============================================

\usepackage{listings}
\usepackage{xcolor}

%% Define colors for syntax highlighting
\definecolor{codegreen}{rgb}{0,0.6,0}
\definecolor{codegray}{rgb}{0.5,0.5,0.5}
\definecolor{codepurple}{rgb}{0.58,0,0.82}
\definecolor{codeblue}{rgb}{0.13,0.29,0.53}
\definecolor{codebg}{rgb}{0.95,0.95,0.95}

%% Configure listings style
\lstdefinestyle{default}{
    backgroundcolor=\color{codebg},
    commentstyle=\color{codegreen},
    keywordstyle=\color{codeblue}\bfseries,
    numberstyle=\tiny\color{codegray},
    stringstyle=\color{codepurple},
    basicstyle=\ttfamily\footnotesize,
    breakatwhitespace=false,
    breaklines=true,
    captionpos=b,
    keepspaces=true,
    numbers=left,
    numbersep=8pt,
    showspaces=false,
    showstringspaces=false,
    showtabs=false,
    tabsize=2,
    frame=single,
    rulecolor=\color{codegray}
}

\lstset{style=default}

%% Inline code command
\newcommand{\inlinecode}[1]{%
  \colorbox{codebg}{\texttt{\small #1}}%
}

%% Command: Code block with language
%% Usage: In document, use \begin{lstlisting}[language=Python]
%% Or use the custom environment below

%% Custom code environment
\lstnewenvironment{code}[1][]{%
  \lstset{
    style=default,
    #1
  }
}{}

%% Example usage (commented out):
% \inlinecode{print("Hello, World!")}
%
% \begin{code}[language=Python, caption=Example Python Code]
% def hello_world():
%     print("Hello, World!")
% \end{code}

%% ============================================

\usepackage{listings}
\usepackage{xcolor}

%% Define colors for syntax highlighting
\definecolor{codegreen}{rgb}{0,0.6,0}
\definecolor{codegray}{rgb}{0.5,0.5,0.5}
\definecolor{codepurple}{rgb}{0.58,0,0.82}
\definecolor{codeblue}{rgb}{0.13,0.29,0.53}
\definecolor{codebg}{rgb}{0.95,0.95,0.95}

%% Configure listings style
\lstdefinestyle{default}{
    backgroundcolor=\color{codebg},
    commentstyle=\color{codegreen},
    keywordstyle=\color{codeblue}\bfseries,
    numberstyle=\tiny\color{codegray},
    stringstyle=\color{codepurple},
    basicstyle=\ttfamily\footnotesize,
    breakatwhitespace=false,
    breaklines=true,
    captionpos=b,
    keepspaces=true,
    numbers=left,
    numbersep=8pt,
    showspaces=false,
    showstringspaces=false,
    showtabs=false,
    tabsize=2,
    frame=single,
    rulecolor=\color{codegray}
}

\lstset{style=default}

%% Inline code command
\newcommand{\inlinecode}[1]{%
  \colorbox{codebg}{\texttt{\small #1}}%
}

%% Command: Code block with language
%% Usage: In document, use \begin{lstlisting}[language=Python]
%% Or use the custom environment below

%% Custom code environment
\lstnewenvironment{code}[1][]{%
  \lstset{
    style=default,
    #1
  }
}{}

%% Example usage (commented out):
% \inlinecode{print("Hello, World!")}
%
% \begin{code}[language=Python, caption=Example Python Code]
% def hello_world():
%     print("Hello, World!")
% \end{code}
